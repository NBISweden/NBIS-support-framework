\documentclass[10pt,a4paper,]{article}
\usepackage{lmodern}
\usepackage{amssymb,amsmath}
\usepackage{ifxetex,ifluatex}
\usepackage{fixltx2e} % provides \textsubscript
\ifnum 0\ifxetex 1\fi\ifluatex 1\fi=0 % if pdftex
  \usepackage[T1]{fontenc}
  \usepackage[utf8]{inputenc}
\else % if luatex or xelatex
  \ifxetex
    \usepackage{mathspec}
  \else
    \usepackage{fontspec}
  \fi
  \defaultfontfeatures{Ligatures=TeX,Scale=MatchLowercase}
\fi
% use upquote if available, for straight quotes in verbatim environments
\IfFileExists{upquote.sty}{\usepackage{upquote}}{}
% use microtype if available
\IfFileExists{microtype.sty}{%
\usepackage{microtype}
\UseMicrotypeSet[protrusion]{basicmath} % disable protrusion for tt fonts
}{}
\usepackage[margin=1in]{geometry}
\usepackage{hyperref}
\PassOptionsToPackage{usenames,dvipsnames}{color} % color is loaded by hyperref
\hypersetup{unicode=true,
            pdftitle={NBIS: document},
            colorlinks=true,
            linkcolor=Maroon,
            citecolor=Blue,
            urlcolor=blue,
            breaklinks=true}
\urlstyle{same}  % don't use monospace font for urls
\usepackage{natbib}
\bibliographystyle{plainnat}
\usepackage{graphicx,grffile}
\makeatletter
\def\maxwidth{\ifdim\Gin@nat@width>\linewidth\linewidth\else\Gin@nat@width\fi}
\def\maxheight{\ifdim\Gin@nat@height>\textheight\textheight\else\Gin@nat@height\fi}
\makeatother
% Scale images if necessary, so that they will not overflow the page
% margins by default, and it is still possible to overwrite the defaults
% using explicit options in \includegraphics[width, height, ...]{}
\setkeys{Gin}{width=\maxwidth,height=\maxheight,keepaspectratio}
\IfFileExists{parskip.sty}{%
\usepackage{parskip}
}{% else
\setlength{\parindent}{0pt}
\setlength{\parskip}{6pt plus 2pt minus 1pt}
}
\setlength{\emergencystretch}{3em}  % prevent overfull lines
\providecommand{\tightlist}{%
  \setlength{\itemsep}{0pt}\setlength{\parskip}{0pt}}
\setcounter{secnumdepth}{0}
% Redefines (sub)paragraphs to behave more like sections
\ifx\paragraph\undefined\else
\let\oldparagraph\paragraph
\renewcommand{\paragraph}[1]{\oldparagraph{#1}\mbox{}}
\fi
\ifx\subparagraph\undefined\else
\let\oldsubparagraph\subparagraph
\renewcommand{\subparagraph}[1]{\oldsubparagraph{#1}\mbox{}}
\fi

%%% Use protect on footnotes to avoid problems with footnotes in titles
\let\rmarkdownfootnote\footnote%
\def\footnote{\protect\rmarkdownfootnote}

%%% Change title format to be more compact
\usepackage{titling}

% Create subtitle command for use in maketitle
\newcommand{\subtitle}[1]{
  \posttitle{
    \begin{center}\large#1\end{center}
    }
}

\setlength{\droptitle}{-2em}

  \title{NBIS: document}
    \pretitle{\vspace{\droptitle}\centering\huge}
  \posttitle{\par}
  \subtitle{Data management plan for support projects}
  \author{}
    \preauthor{}\postauthor{}
    \date{}
    \predate{}\postdate{}
  
\usepackage{fancyhdr}
\usepackage{lastpage}


% Defining header
\addtolength{\headheight}{2.5cm} % make more space for the header
\pagestyle{fancyplain} % use fancy for all pages except chapter start
\lhead{\includegraphics[height=1.4cm, width=2.2cm]{NBIS-logo.png}} % left logo
\rhead{\includegraphics[height=1.4cm, width=4cm]{SciLifeLab-logo.jpg}} % right logo
%\rhead{\today}
\addtolength{\headheight}{-2.5cm} % make more space for the header
\cfoot{\thepage\ (\pageref{LastPage})}
\lfoot{Org number: 202100-2932}
\rfoot{https://nbis.se}
\renewcommand{\headrulewidth}{0pt} % remove rule below header

\begin{document}
\maketitle

\subsection{Introduction}\label{introduction}

Research data management plans (DMP) has become a crucial part of life
science research. Good data management practices help to organised,
analyse, store, publish and re-use data. DMPs are not only a good
practice, but may and will be required to consider and document under
the policy of various research founders, e.g.~VR. This document contains
some key points to consider when having data analysed with NBIS.

\subsection{Overview}\label{overview}

\begin{enumerate}
\def\labelenumi{\arabic{enumi}.}
\tightlist
\item
  Project title
\item
  Project issue number
\item
  Project contact person
\item
  Data contact person
\item
  Data management contact person
\item
  Is DMP available? Yes/No. If yes, provide details. If no, provide the
  information below. Refer to
  \href{https://docs.google.com/document/d/1g6vJNIrkSnylASkNHB9Zwm5N6jvTgoSxBjS_bexRPsY/edit\#heading=h.y6r21qqu4ir4}{extended
  online version of this document} for sections examples and more
  information.
\end{enumerate}

\subsection{Data description}\label{data-description}

\begin{enumerate}
\def\labelenumi{\arabic{enumi}.}
\tightlist
\item
  Primary data, \emph{incl. a) what type of data will be generated? b)
  from which type of samples? c) how many samples? d) from what
  technical platform?}
\item
  Additional data, \emph{incl. a) what other data will you need to
  perform the project?, b) do you have the necessary permission(s) to
  use that data? c) how does your re-use complies with terms and
  conditions?}
\end{enumerate}

\subsection{Ethical \& Legal aspects}\label{ethical-legal-aspects}

\begin{enumerate}
\def\labelenumi{\arabic{enumi}.}
\tightlist
\item
  Data owner(s), \emph{incl. a) which host university/universities owns
  the data?}
\item
  Intellectual property rights (IPR)/Copyrights, \emph{incl. a) are
  there IPR or copyright issues to consider? b) will the data
  potentially be used to generate IPR?}
\item
  Legal Agreements (if applicable), \emph{incl a) what are the
  agreements with other stakeholders? b) what agreements are needed to
  regulate relationships between collaborators. c) do you need Data
  Processing Agreements between collaborators for personal data?}
\item
  Sensitive Human Data (if applicable), \emph{incl. a) does your project
  have approval by an ethics committee?, b) what informed consents exist
  for the samples?, c) are they any consent codes that can be assigned
  to the samples?}
\end{enumerate}

\subsection{Data documentation}\label{data-documentation}

\begin{enumerate}
\def\labelenumi{\arabic{enumi}.}
\tightlist
\item
  Sample metadata documentation, \emph{incl. a) what metadata will be
  provided with the collected/generated/reused data? b) are there
  metadata standards that you can use?}
\item
  Dataset documentation, \emph{incl. a) which file formats and data
  types will be used for the data?, b) what is the estimated total size
  of the data?}
\end{enumerate}

\subsection{Data storage \& backup}\label{data-storage-backup}

\begin{enumerate}
\def\labelenumi{\arabic{enumi}.}
\tightlist
\item
  Storage, \emph{incl. a) how, b) where, and c) for how long will the
  data be stored during the analysis phase of the project?}
\item
  Backup, \emph{incl. a) how, b) where and c) at what intervals will the
  data be backed-up? d) how will data be recovered in the case of a data
  loss incident?}
\item
  Security, \emph{incl. a) how will you ensure that only authorized
  persons have access to the data?, b) how will sensitive data be
  protected, if applicable?}
\end{enumerate}

\subsection{Data publication \&
archiving}\label{data-publication-archiving}

\begin{enumerate}
\def\labelenumi{\arabic{enumi}.}
\tightlist
\item
  Long-term storage, \emph{incl. a) how and b) where will the data be
  stored after the project's completion?}
\item
  Data publishing, \emph{incl. a) will you deposit your data to a
  trusted data repository? b) if so, when during the project will you
  submit the data to the archive? c) in what formats will you submit the
  data to the repository?, d) will your data receive a persistent
  identifier?}
\item
  Data access, \emph{incl. a) will your data be available Open or
  Controlled Access? b) when will the data become accessible? c) if
  Controlled Access, who controls access?, d) will all data and
  metadata, or only parts of it, be published?, e) under what licence(s)
  or terms will you share your data and code?, f) Are there any
  restrictions that prevents the publication of all the material? and g)
  if so, what actions must be taken before the material can be made
  available?}
\end{enumerate}

\subsection{Costs}\label{costs}

\begin{enumerate}
\def\labelenumi{\arabic{enumi}.}
\tightlist
\item
  Are there costs you need to consider to buy and manage specific
  software or hardware?
\item
  What are the costs you need to consider for storage and backup?
\end{enumerate}

\begin{center}\rule{0.5\linewidth}{\linethickness}\end{center}

\subsection{Ultra-short general
recommendations}\label{ultra-short-general-recommendations}

\begin{enumerate}
\def\labelenumi{\arabic{enumi}.}
\tightlist
\item
  If you have no specific insight, include a data management and
  archiving cost of 5-10\% of the project budget for a large-scale omics
  project, provided that you can largely rely on SNIC systems. This does
  not include any personnel cost to analyse the data.
\item
  If you cannot use SNIC systems for your data analysis/storage, you
  need a major budget post to cover for this, and you should investigate
  this carefully.
\item
  Human sequencing data are classified as sensitive data under GDPR, and
  needs special attention. More information under
  \url{https://nbis.se/support/human-data.html}
\item
  Submit data to public repositories early in the project, i.e.~under
  embargo, to ensure an extra backup.
\end{enumerate}

\begin{center}\rule{0.5\linewidth}{\linethickness}\end{center}

\subsection{More information}\label{more-information}

\begin{itemize}
\tightlist
\item
  Extension of this document with examples and including
  \href{https://docs.google.com/document/d/1g6vJNIrkSnylASkNHB9Zwm5N6jvTgoSxBjS_bexRPsY/edit\#heading=h.wvltu9dpdsqz}{DMP
  knowlege hub}
\item
  SNIC resources for compute and storage during active phase of the
  project \url{http://snic.se}
\item
  Working with sensitive data:
  \url{https://nbis.se/support/human-data.html}
\item
  A list of public deposition databases can be found on
  \href{https://www.elixir-europe.org/platforms/data/elixir-deposition-databases}{Elixir
  Deposition Databases List}
\item
  A Quick Guide to Organizing Computational Biology Projects
  \citep{Noble2009} 
\end{itemize}

\bibliography{bibliography.bib}


\end{document}
