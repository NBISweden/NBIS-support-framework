

% A latex document created by ipypublish
% outline: ipypublish.templates.outline_schemas/latex_outline.latex.j2
% with segments:
% - standard-standard_packages: with standard nbconvert packages
% - standard-standard_definitions: with standard nbconvert definitions
% - ipypublish-doc_article: with the main ipypublish article setup
% - ipypublish-front_pages: with the main ipypublish title and contents page setup
% - ipypublish-biblio_natbib: with the main ipypublish bibliography
% - ipypublish-contents_output: with the main ipypublish content
% - ipypublish-contents_framed_code: with the input code wrapped and framed
%
%%%%%%%%%%%% DOCCLASS

\documentclass[10pt,parskip=half,
toc=sectionentrywithdots,
bibliography=totocnumbered,
captions=tableheading,numbers=noendperiod]{scrartcl}
%\usepackage{polyglossia}
%\setmainlanguage{british}
%\DeclareTextCommandDefault{\nobreakspace}{\leavevmode\nobreak\ }
\usepackage[british]{babel}

%%%%%%%%%%%%

%%%%%%%%%%%% PACKAGES

\usepackage[T1]{fontenc} % Nicer default font (+ math font) than Computer Modern for most use cases
\usepackage{mathpazo}
\usepackage{graphicx}
\usepackage[skip=3pt]{caption}
\usepackage{adjustbox} % Used to constrain images to a maximum size
\usepackage[table]{xcolor} % Allow colors to be defined
\usepackage{enumerate} % Needed for markdown enumerations to work
\usepackage{amsmath} % Equations
\usepackage{amssymb} % Equations
\usepackage{textcomp} % defines textquotesingle
% Hack from http://tex.stackexchange.com/a/47451/13684:
\AtBeginDocument{%
    \def\PYZsq{\textquotesingle}% Upright quotes in Pygmentized code
}
\usepackage{upquote} % Upright quotes for verbatim code
\usepackage{eurosym} % defines \euro
\usepackage[mathletters]{ucs} % Extended unicode (utf-8) support
\usepackage[utf8x]{inputenc} % Allow utf-8 characters in the tex document
\usepackage{fancyvrb} % verbatim replacement that allows latex
\usepackage{grffile} % extends the file name processing of package graphics
                        % to support a larger range
% The hyperref package gives us a pdf with properly built
% internal navigation ('pdf bookmarks' for the table of contents,
% internal cross-reference links, web links for URLs, etc.)
\usepackage{hyperref}
\usepackage{longtable} % longtable support required by pandoc >1.10
\usepackage{booktabs}  % table support for pandoc > 1.12.2
\usepackage[inline]{enumitem} % IRkernel/repr support (it uses the enumerate* environment)
\usepackage[normalem]{ulem} % ulem is needed to support strikethroughs (\sout)
                            % normalem makes italics be italics, not underlines

\usepackage{translations}
\usepackage{microtype} % improves the spacing between words and letters
\usepackage{placeins} % placement of figures
% could use \usepackage[section]{placeins} but placing in subsection in command section
% Places the float at precisely the location in the LaTeX code (with H)
\usepackage{float}
\usepackage[colorinlistoftodos,obeyFinal,textwidth=.8in]{todonotes} % to mark to-dos
% number figures, tables and equations by section
% fix for new versions of texlive (see https://tex.stackexchange.com/a/425603/107738)
\let\counterwithout\relax
\let\counterwithin\relax
\usepackage{chngcntr}
% header/footer
\usepackage[footsepline=0.25pt]{scrlayer-scrpage}

% bibliography formatting
\usepackage[numbers, square, super, sort&compress]{natbib}
% hyperlink doi's
\usepackage{doi}

    % define a code float
    \usepackage{newfloat} % to define a new float types
    \DeclareFloatingEnvironment[
        fileext=frm,placement={!ht},
        within=section,name=Code]{codecell}
    \DeclareFloatingEnvironment[
        fileext=frm,placement={!ht},
        within=section,name=Text]{textcell}
    \DeclareFloatingEnvironment[
        fileext=frm,placement={!ht},
        within=section,name=Text]{errorcell}

    \usepackage{listings} % a package for wrapping code in a box
    \usepackage[framemethod=tikz]{mdframed} % to fram code

%%%%%%%%%%%%

%%%%%%%%%%%% DEFINITIONS

% Pygments definitions

\makeatletter
\def\PY@reset{\let\PY@it=\relax \let\PY@bf=\relax%
    \let\PY@ul=\relax \let\PY@tc=\relax%
    \let\PY@bc=\relax \let\PY@ff=\relax}
\def\PY@tok#1{\csname PY@tok@#1\endcsname}
\def\PY@toks#1+{\ifx\relax#1\empty\else%
    \PY@tok{#1}\expandafter\PY@toks\fi}
\def\PY@do#1{\PY@bc{\PY@tc{\PY@ul{%
    \PY@it{\PY@bf{\PY@ff{#1}}}}}}}
\def\PY#1#2{\PY@reset\PY@toks#1+\relax+\PY@do{#2}}

\expandafter\def\csname PY@tok@w\endcsname{\def\PY@tc##1{\textcolor[rgb]{0.73,0.73,0.73}{##1}}}
\expandafter\def\csname PY@tok@c\endcsname{\let\PY@it=\textit\def\PY@tc##1{\textcolor[rgb]{0.25,0.50,0.50}{##1}}}
\expandafter\def\csname PY@tok@cp\endcsname{\def\PY@tc##1{\textcolor[rgb]{0.74,0.48,0.00}{##1}}}
\expandafter\def\csname PY@tok@k\endcsname{\let\PY@bf=\textbf\def\PY@tc##1{\textcolor[rgb]{0.00,0.50,0.00}{##1}}}
\expandafter\def\csname PY@tok@kp\endcsname{\def\PY@tc##1{\textcolor[rgb]{0.00,0.50,0.00}{##1}}}
\expandafter\def\csname PY@tok@kt\endcsname{\def\PY@tc##1{\textcolor[rgb]{0.69,0.00,0.25}{##1}}}
\expandafter\def\csname PY@tok@o\endcsname{\def\PY@tc##1{\textcolor[rgb]{0.40,0.40,0.40}{##1}}}
\expandafter\def\csname PY@tok@ow\endcsname{\let\PY@bf=\textbf\def\PY@tc##1{\textcolor[rgb]{0.67,0.13,1.00}{##1}}}
\expandafter\def\csname PY@tok@nb\endcsname{\def\PY@tc##1{\textcolor[rgb]{0.00,0.50,0.00}{##1}}}
\expandafter\def\csname PY@tok@nf\endcsname{\def\PY@tc##1{\textcolor[rgb]{0.00,0.00,1.00}{##1}}}
\expandafter\def\csname PY@tok@nc\endcsname{\let\PY@bf=\textbf\def\PY@tc##1{\textcolor[rgb]{0.00,0.00,1.00}{##1}}}
\expandafter\def\csname PY@tok@nn\endcsname{\let\PY@bf=\textbf\def\PY@tc##1{\textcolor[rgb]{0.00,0.00,1.00}{##1}}}
\expandafter\def\csname PY@tok@ne\endcsname{\let\PY@bf=\textbf\def\PY@tc##1{\textcolor[rgb]{0.82,0.25,0.23}{##1}}}
\expandafter\def\csname PY@tok@nv\endcsname{\def\PY@tc##1{\textcolor[rgb]{0.10,0.09,0.49}{##1}}}
\expandafter\def\csname PY@tok@no\endcsname{\def\PY@tc##1{\textcolor[rgb]{0.53,0.00,0.00}{##1}}}
\expandafter\def\csname PY@tok@nl\endcsname{\def\PY@tc##1{\textcolor[rgb]{0.63,0.63,0.00}{##1}}}
\expandafter\def\csname PY@tok@ni\endcsname{\let\PY@bf=\textbf\def\PY@tc##1{\textcolor[rgb]{0.60,0.60,0.60}{##1}}}
\expandafter\def\csname PY@tok@na\endcsname{\def\PY@tc##1{\textcolor[rgb]{0.49,0.56,0.16}{##1}}}
\expandafter\def\csname PY@tok@nt\endcsname{\let\PY@bf=\textbf\def\PY@tc##1{\textcolor[rgb]{0.00,0.50,0.00}{##1}}}
\expandafter\def\csname PY@tok@nd\endcsname{\def\PY@tc##1{\textcolor[rgb]{0.67,0.13,1.00}{##1}}}
\expandafter\def\csname PY@tok@s\endcsname{\def\PY@tc##1{\textcolor[rgb]{0.73,0.13,0.13}{##1}}}
\expandafter\def\csname PY@tok@sd\endcsname{\let\PY@it=\textit\def\PY@tc##1{\textcolor[rgb]{0.73,0.13,0.13}{##1}}}
\expandafter\def\csname PY@tok@si\endcsname{\let\PY@bf=\textbf\def\PY@tc##1{\textcolor[rgb]{0.73,0.40,0.53}{##1}}}
\expandafter\def\csname PY@tok@se\endcsname{\let\PY@bf=\textbf\def\PY@tc##1{\textcolor[rgb]{0.73,0.40,0.13}{##1}}}
\expandafter\def\csname PY@tok@sr\endcsname{\def\PY@tc##1{\textcolor[rgb]{0.73,0.40,0.53}{##1}}}
\expandafter\def\csname PY@tok@ss\endcsname{\def\PY@tc##1{\textcolor[rgb]{0.10,0.09,0.49}{##1}}}
\expandafter\def\csname PY@tok@sx\endcsname{\def\PY@tc##1{\textcolor[rgb]{0.00,0.50,0.00}{##1}}}
\expandafter\def\csname PY@tok@m\endcsname{\def\PY@tc##1{\textcolor[rgb]{0.40,0.40,0.40}{##1}}}
\expandafter\def\csname PY@tok@gh\endcsname{\let\PY@bf=\textbf\def\PY@tc##1{\textcolor[rgb]{0.00,0.00,0.50}{##1}}}
\expandafter\def\csname PY@tok@gu\endcsname{\let\PY@bf=\textbf\def\PY@tc##1{\textcolor[rgb]{0.50,0.00,0.50}{##1}}}
\expandafter\def\csname PY@tok@gd\endcsname{\def\PY@tc##1{\textcolor[rgb]{0.63,0.00,0.00}{##1}}}
\expandafter\def\csname PY@tok@gi\endcsname{\def\PY@tc##1{\textcolor[rgb]{0.00,0.63,0.00}{##1}}}
\expandafter\def\csname PY@tok@gr\endcsname{\def\PY@tc##1{\textcolor[rgb]{1.00,0.00,0.00}{##1}}}
\expandafter\def\csname PY@tok@ge\endcsname{\let\PY@it=\textit}
\expandafter\def\csname PY@tok@gs\endcsname{\let\PY@bf=\textbf}
\expandafter\def\csname PY@tok@gp\endcsname{\let\PY@bf=\textbf\def\PY@tc##1{\textcolor[rgb]{0.00,0.00,0.50}{##1}}}
\expandafter\def\csname PY@tok@go\endcsname{\def\PY@tc##1{\textcolor[rgb]{0.53,0.53,0.53}{##1}}}
\expandafter\def\csname PY@tok@gt\endcsname{\def\PY@tc##1{\textcolor[rgb]{0.00,0.27,0.87}{##1}}}
\expandafter\def\csname PY@tok@err\endcsname{\def\PY@bc##1{\setlength{\fboxsep}{0pt}\fcolorbox[rgb]{1.00,0.00,0.00}{1,1,1}{\strut ##1}}}
\expandafter\def\csname PY@tok@kc\endcsname{\let\PY@bf=\textbf\def\PY@tc##1{\textcolor[rgb]{0.00,0.50,0.00}{##1}}}
\expandafter\def\csname PY@tok@kd\endcsname{\let\PY@bf=\textbf\def\PY@tc##1{\textcolor[rgb]{0.00,0.50,0.00}{##1}}}
\expandafter\def\csname PY@tok@kn\endcsname{\let\PY@bf=\textbf\def\PY@tc##1{\textcolor[rgb]{0.00,0.50,0.00}{##1}}}
\expandafter\def\csname PY@tok@kr\endcsname{\let\PY@bf=\textbf\def\PY@tc##1{\textcolor[rgb]{0.00,0.50,0.00}{##1}}}
\expandafter\def\csname PY@tok@bp\endcsname{\def\PY@tc##1{\textcolor[rgb]{0.00,0.50,0.00}{##1}}}
\expandafter\def\csname PY@tok@fm\endcsname{\def\PY@tc##1{\textcolor[rgb]{0.00,0.00,1.00}{##1}}}
\expandafter\def\csname PY@tok@vc\endcsname{\def\PY@tc##1{\textcolor[rgb]{0.10,0.09,0.49}{##1}}}
\expandafter\def\csname PY@tok@vg\endcsname{\def\PY@tc##1{\textcolor[rgb]{0.10,0.09,0.49}{##1}}}
\expandafter\def\csname PY@tok@vi\endcsname{\def\PY@tc##1{\textcolor[rgb]{0.10,0.09,0.49}{##1}}}
\expandafter\def\csname PY@tok@vm\endcsname{\def\PY@tc##1{\textcolor[rgb]{0.10,0.09,0.49}{##1}}}
\expandafter\def\csname PY@tok@sa\endcsname{\def\PY@tc##1{\textcolor[rgb]{0.73,0.13,0.13}{##1}}}
\expandafter\def\csname PY@tok@sb\endcsname{\def\PY@tc##1{\textcolor[rgb]{0.73,0.13,0.13}{##1}}}
\expandafter\def\csname PY@tok@sc\endcsname{\def\PY@tc##1{\textcolor[rgb]{0.73,0.13,0.13}{##1}}}
\expandafter\def\csname PY@tok@dl\endcsname{\def\PY@tc##1{\textcolor[rgb]{0.73,0.13,0.13}{##1}}}
\expandafter\def\csname PY@tok@s2\endcsname{\def\PY@tc##1{\textcolor[rgb]{0.73,0.13,0.13}{##1}}}
\expandafter\def\csname PY@tok@sh\endcsname{\def\PY@tc##1{\textcolor[rgb]{0.73,0.13,0.13}{##1}}}
\expandafter\def\csname PY@tok@s1\endcsname{\def\PY@tc##1{\textcolor[rgb]{0.73,0.13,0.13}{##1}}}
\expandafter\def\csname PY@tok@mb\endcsname{\def\PY@tc##1{\textcolor[rgb]{0.40,0.40,0.40}{##1}}}
\expandafter\def\csname PY@tok@mf\endcsname{\def\PY@tc##1{\textcolor[rgb]{0.40,0.40,0.40}{##1}}}
\expandafter\def\csname PY@tok@mh\endcsname{\def\PY@tc##1{\textcolor[rgb]{0.40,0.40,0.40}{##1}}}
\expandafter\def\csname PY@tok@mi\endcsname{\def\PY@tc##1{\textcolor[rgb]{0.40,0.40,0.40}{##1}}}
\expandafter\def\csname PY@tok@il\endcsname{\def\PY@tc##1{\textcolor[rgb]{0.40,0.40,0.40}{##1}}}
\expandafter\def\csname PY@tok@mo\endcsname{\def\PY@tc##1{\textcolor[rgb]{0.40,0.40,0.40}{##1}}}
\expandafter\def\csname PY@tok@ch\endcsname{\let\PY@it=\textit\def\PY@tc##1{\textcolor[rgb]{0.25,0.50,0.50}{##1}}}
\expandafter\def\csname PY@tok@cm\endcsname{\let\PY@it=\textit\def\PY@tc##1{\textcolor[rgb]{0.25,0.50,0.50}{##1}}}
\expandafter\def\csname PY@tok@cpf\endcsname{\let\PY@it=\textit\def\PY@tc##1{\textcolor[rgb]{0.25,0.50,0.50}{##1}}}
\expandafter\def\csname PY@tok@c1\endcsname{\let\PY@it=\textit\def\PY@tc##1{\textcolor[rgb]{0.25,0.50,0.50}{##1}}}
\expandafter\def\csname PY@tok@cs\endcsname{\let\PY@it=\textit\def\PY@tc##1{\textcolor[rgb]{0.25,0.50,0.50}{##1}}}

\def\PYZbs{\char`\\}
\def\PYZus{\char`\_}
\def\PYZob{\char`\{}
\def\PYZcb{\char`\}}
\def\PYZca{\char`\^}
\def\PYZam{\char`\&}
\def\PYZlt{\char`\<}
\def\PYZgt{\char`\>}
\def\PYZsh{\char`\#}
\def\PYZpc{\char`\%}
\def\PYZdl{\char`\$}
\def\PYZhy{\char`\-}
\def\PYZsq{\char`\'}
\def\PYZdq{\char`\"}
\def\PYZti{\char`\~}
% for compatibility with earlier versions
\def\PYZat{@}
\def\PYZlb{[}
\def\PYZrb{]}
\makeatother

% ANSI colors
\definecolor{ansi-black}{HTML}{3E424D}
\definecolor{ansi-black-intense}{HTML}{282C36}
\definecolor{ansi-red}{HTML}{E75C58}
\definecolor{ansi-red-intense}{HTML}{B22B31}
\definecolor{ansi-green}{HTML}{00A250}
\definecolor{ansi-green-intense}{HTML}{007427}
\definecolor{ansi-yellow}{HTML}{DDB62B}
\definecolor{ansi-yellow-intense}{HTML}{B27D12}
\definecolor{ansi-blue}{HTML}{208FFB}
\definecolor{ansi-blue-intense}{HTML}{0065CA}
\definecolor{ansi-magenta}{HTML}{D160C4}
\definecolor{ansi-magenta-intense}{HTML}{A03196}
\definecolor{ansi-cyan}{HTML}{60C6C8}
\definecolor{ansi-cyan-intense}{HTML}{258F8F}
\definecolor{ansi-white}{HTML}{C5C1B4}
\definecolor{ansi-white-intense}{HTML}{A1A6B2}

% commands and environments needed by pandoc snippets
% extracted from the output of `pandoc -s`
\providecommand{\tightlist}{%
  \setlength{\itemsep}{0pt}\setlength{\parskip}{0pt}}
\DefineVerbatimEnvironment{Highlighting}{Verbatim}{commandchars=\\\{\}}
% Add ',fontsize=\small' for more characters per line
\newenvironment{Shaded}{}{}
\newcommand{\KeywordTok}[1]{\textcolor[rgb]{0.00,0.44,0.13}{\textbf{{#1}}}}
\newcommand{\DataTypeTok}[1]{\textcolor[rgb]{0.56,0.13,0.00}{{#1}}}
\newcommand{\DecValTok}[1]{\textcolor[rgb]{0.25,0.63,0.44}{{#1}}}
\newcommand{\BaseNTok}[1]{\textcolor[rgb]{0.25,0.63,0.44}{{#1}}}
\newcommand{\FloatTok}[1]{\textcolor[rgb]{0.25,0.63,0.44}{{#1}}}
\newcommand{\CharTok}[1]{\textcolor[rgb]{0.25,0.44,0.63}{{#1}}}
\newcommand{\StringTok}[1]{\textcolor[rgb]{0.25,0.44,0.63}{{#1}}}
\newcommand{\CommentTok}[1]{\textcolor[rgb]{0.38,0.63,0.69}{\textit{{#1}}}}
\newcommand{\OtherTok}[1]{\textcolor[rgb]{0.00,0.44,0.13}{{#1}}}
\newcommand{\AlertTok}[1]{\textcolor[rgb]{1.00,0.00,0.00}{\textbf{{#1}}}}
\newcommand{\FunctionTok}[1]{\textcolor[rgb]{0.02,0.16,0.49}{{#1}}}
\newcommand{\RegionMarkerTok}[1]{{#1}}
\newcommand{\ErrorTok}[1]{\textcolor[rgb]{1.00,0.00,0.00}{\textbf{{#1}}}}
\newcommand{\NormalTok}[1]{{#1}}

% Additional commands for more recent versions of Pandoc
\newcommand{\ConstantTok}[1]{\textcolor[rgb]{0.53,0.00,0.00}{{#1}}}
\newcommand{\SpecialCharTok}[1]{\textcolor[rgb]{0.25,0.44,0.63}{{#1}}}
\newcommand{\VerbatimStringTok}[1]{\textcolor[rgb]{0.25,0.44,0.63}{{#1}}}
\newcommand{\SpecialStringTok}[1]{\textcolor[rgb]{0.73,0.40,0.53}{{#1}}}
\newcommand{\ImportTok}[1]{{#1}}
\newcommand{\DocumentationTok}[1]{\textcolor[rgb]{0.73,0.13,0.13}{\textit{{#1}}}}
\newcommand{\AnnotationTok}[1]{\textcolor[rgb]{0.38,0.63,0.69}{\textbf{\textit{{#1}}}}}
\newcommand{\CommentVarTok}[1]{\textcolor[rgb]{0.38,0.63,0.69}{\textbf{\textit{{#1}}}}}
\newcommand{\VariableTok}[1]{\textcolor[rgb]{0.10,0.09,0.49}{{#1}}}
\newcommand{\ControlFlowTok}[1]{\textcolor[rgb]{0.00,0.44,0.13}{\textbf{{#1}}}}
\newcommand{\OperatorTok}[1]{\textcolor[rgb]{0.40,0.40,0.40}{{#1}}}
\newcommand{\BuiltInTok}[1]{{#1}}
\newcommand{\ExtensionTok}[1]{{#1}}
\newcommand{\PreprocessorTok}[1]{\textcolor[rgb]{0.74,0.48,0.00}{{#1}}}
\newcommand{\AttributeTok}[1]{\textcolor[rgb]{0.49,0.56,0.16}{{#1}}}
\newcommand{\InformationTok}[1]{\textcolor[rgb]{0.38,0.63,0.69}{\textbf{\textit{{#1}}}}}
\newcommand{\WarningTok}[1]{\textcolor[rgb]{0.38,0.63,0.69}{\textbf{\textit{{#1}}}}}

% Define a nice break command that doesn't care if a line doesn't already
% exist.
\def\br{\hspace*{\fill} \\* }

% Math Jax compatability definitions
\def\gt{>}
\def\lt{<}

\setcounter{secnumdepth}{5}

% Colors for the hyperref package
\definecolor{urlcolor}{rgb}{0,.145,.698}
\definecolor{linkcolor}{rgb}{.71,0.21,0.01}
\definecolor{citecolor}{rgb}{.12,.54,.11}

\DeclareTranslationFallback{Author}{Author}
\DeclareTranslation{Portuges}{Author}{Autor}

\DeclareTranslationFallback{List of Codes}{List of Codes}
\DeclareTranslation{Catalan}{List of Codes}{Llista de Codis}
\DeclareTranslation{Danish}{List of Codes}{Liste over Koder}
\DeclareTranslation{German}{List of Codes}{Liste der Codes}
\DeclareTranslation{Spanish}{List of Codes}{Lista de C\'{o}digos}
\DeclareTranslation{French}{List of Codes}{Liste des Codes}
\DeclareTranslation{Italian}{List of Codes}{Elenco dei Codici}
\DeclareTranslation{Dutch}{List of Codes}{Lijst van Codes}
\DeclareTranslation{Portuges}{List of Codes}{Lista de C\'{o}digos}

\DeclareTranslationFallback{Supervisors}{Supervisors}
\DeclareTranslation{Catalan}{Supervisors}{Supervisors}
\DeclareTranslation{Danish}{Supervisors}{Vejledere}
\DeclareTranslation{German}{Supervisors}{Vorgesetzten}
\DeclareTranslation{Spanish}{Supervisors}{Supervisores}
\DeclareTranslation{French}{Supervisors}{Superviseurs}
\DeclareTranslation{Italian}{Supervisors}{Le autorit\`{a} di vigilanza}
\DeclareTranslation{Dutch}{Supervisors}{supervisors}
\DeclareTranslation{Portuguese}{Supervisors}{Supervisores}

\definecolor{codegreen}{rgb}{0,0.6,0}
\definecolor{codegray}{rgb}{0.5,0.5,0.5}
\definecolor{codepurple}{rgb}{0.58,0,0.82}
\definecolor{backcolour}{rgb}{0.95,0.95,0.95}

\lstdefinestyle{mystyle}{
    commentstyle=\color{codegreen},
    keywordstyle=\color{magenta},
    numberstyle=\tiny\color{codegray},
    stringstyle=\color{codepurple},
    basicstyle=\ttfamily,
    breakatwhitespace=false,
    keepspaces=true,
    numbers=left,
    numbersep=10pt,
    showspaces=false,
    showstringspaces=false,
    showtabs=false,
    tabsize=2,
    breaklines=true,
    literate={\-}{}{0\discretionary{-}{}{-}},
  postbreak=\mbox{\textcolor{red}{$\hookrightarrow$}\space},
}

\lstset{style=mystyle}

\surroundwithmdframed[
  hidealllines=true,
  backgroundcolor=backcolour,
  innerleftmargin=0pt,
  innerrightmargin=0pt,
  innertopmargin=0pt,
  innerbottommargin=0pt]{lstlisting}

%%%%%%%%%%%%

%%%%%%%%%%%% MARGINS

 % Used to adjust the document margins
\usepackage{geometry}
\geometry{tmargin=1in,bmargin=1in,lmargin=1in,rmargin=1in,
nohead,includefoot,footskip=25pt}
% you can use showframe option to check the margins visually
%%%%%%%%%%%%

%%%%%%%%%%%% COMMANDS

% ensure new section starts on new page
\addtokomafont{section}{\clearpage}

% Prevent overflowing lines due to hard-to-break entities
\sloppy

% Setup hyperref package
\hypersetup{
    breaklinks=true,  % so long urls are correctly broken across lines
    colorlinks=true,
    urlcolor=urlcolor,
    linkcolor=linkcolor,
    citecolor=citecolor,
    }

% ensure figures are placed within subsections
\makeatletter
\AtBeginDocument{%
    \expandafter\renewcommand\expandafter\subsection\expandafter
    {\expandafter\@fb@secFB\subsection}%
    \newcommand\@fb@secFB{\FloatBarrier
    \gdef\@fb@afterHHook{\@fb@topbarrier \gdef\@fb@afterHHook{}}}%
    \g@addto@macro\@afterheading{\@fb@afterHHook}%
    \gdef\@fb@afterHHook{}%
}
\makeatother

% number figures, tables and equations by section
\counterwithout{figure}{section}
\counterwithout{table}{section}
\counterwithout{equation}{section}
\makeatletter
\@addtoreset{table}{section}
\@addtoreset{figure}{section}
\@addtoreset{equation}{section}
\makeatother
\renewcommand\thetable{\thesection.\arabic{table}}
\renewcommand\thefigure{\thesection.\arabic{figure}}
\renewcommand\theequation{\thesection.\arabic{equation}}

    % set global options for float placement
    \makeatletter
        \providecommand*\setfloatlocations[2]{\@namedef{fps@#1}{#2}}
    \makeatother

% align captions to left (indented)
\captionsetup{justification=raggedright,
singlelinecheck=false,format=hang,labelfont={it,bf}}

% shift footer down so space between separation line
\ModifyLayer[addvoffset=.6ex]{scrheadings.foot.odd}
\ModifyLayer[addvoffset=.6ex]{scrheadings.foot.even}
\ModifyLayer[addvoffset=.6ex]{scrheadings.foot.oneside}
\ModifyLayer[addvoffset=.6ex]{plain.scrheadings.foot.odd}
\ModifyLayer[addvoffset=.6ex]{plain.scrheadings.foot.even}
\ModifyLayer[addvoffset=.6ex]{plain.scrheadings.foot.oneside}
\pagestyle{scrheadings}
\clearscrheadfoot{}
\ifoot{\leftmark}
\renewcommand{\sectionmark}[1]{\markleft{\thesection\ #1}}
\ofoot{\pagemark}
\cfoot{}

%%%%%%%%%%%%

%%%%%%%%%%%% FINAL HEADER MATERIAL

% clereref must be loaded after anything that changes the referencing system
\usepackage{cleveref}
\creflabelformat{equation}{#2#1#3}

% make the code float work with cleverref
\crefname{codecell}{code}{codes}
\Crefname{codecell}{code}{codes}
% make the text float work with cleverref
\crefname{textcell}{text}{texts}
\Crefname{textcell}{text}{texts}
% make the text float work with cleverref
\crefname{errorcell}{error}{errors}
\Crefname{errorcell}{error}{errors}

%%%%%%%%%%%%

\begin{document}

    \begin{titlepage}

  \begin{center}

  \vspace*{1cm}

  \Huge\textbf{Main-Title}

  \vspace{0.5cm}\LARGE{Sub-Title}

  \vspace{1.5cm}

  \begin{minipage}{0.8\textwidth}
    \begin{center}
    \begin{minipage}{0.39\textwidth}
    \begin{flushleft} \Large
    \emph{\GetTranslation{Author}:}\\Authors Name\\\href{mailto:authors@email.com}{authors@email.com}
    \end{flushleft}
    \end{minipage}
    \hspace{\fill}
    \begin{minipage}{0.39\textwidth}
    \begin{flushright} \Large
    \end{flushright}
    \end{minipage}
    \end{center}
  \end{minipage}

  \vfill

  \begin{minipage}{0.8\textwidth}
  \begin{center}\LARGE{A tagline for the report.}
  \end{center}
  \end{minipage}

  \vspace{0.8cm}
      \LARGE{NBIS - National Bioninformatics Infrastructure Sweden}\\
      \LARGE{Science for Life laboratory}\\

  \vspace{0.4cm}

  \today

  \end{center}
  \end{titlepage}

    \begingroup
    \let\cleardoublepage\relax
    \let\clearpage\relax\tableofcontents\listoffigures\listoftables\listof{codecell}{\GetTranslation{List of Codes}}
    \endgroup

\hypertarget{description}{%
\section{Description}\label{description}}

\textbf{Delete this cell before converting to PDF.}

\textbf{Getting Started}\\
This notebook is converted to PDF using
(\href{https://ipypublish.readthedocs.io/en/latest/metadata_tags.html}{ipypublish}).
This requires a LaTeX installation.\\
It is crucial that you show cell metadata, and edit the notebook
metadata.

\textbf{Metadata}\\
There are three levels of metadata: - For notebook level: in the Jupyter
Notebook Toolbar go to Edit -\textgreater{} Edit Notebook Metadata.
\textbf{Edit this for every report}. - For cell level: in the Jupyter
Notebook Toolbar go to View -\textgreater{} Cell Toolbar -\textgreater{}
Edit Metadata and a button will appear above each cell. - For output
level: using
\texttt{IPython.display.display(obj,metadata=\{"ipub":\{\}\})}, you can
set metadata specific to a certain output. Options set at the output
level will override options set at the cell level. For an example of
this, see
\href{https://ipypublish.readthedocs.io/en/latest/code_cells.html\#multiple-outputs}{Multiple
Outputs from a Single Code Cell}. - See
\href{https://ipypublish.readthedocs.io/en/latest/metadata_tags.html}{the
full metadata definitions} to learn how to define different metadata
blocks for your cells.

\textbf{Outputting}\\
To convert to pdf run -
\texttt{nbpublish\ -pdf\ -lb\ Example.ipynb\ -\/-pdf-debug\ -\/-clear-files}
-
\texttt{nbpublish\ -pdf\ -lb\ -f\ latex\_ipypublish\_nocode\ Example.ipynb\ -\/-pdf-debug\ \ -\/-clear-files}
\#hides notebook code unless allowed (see below)\\
This permits merging multiple \texttt{.ipynb} into a single
\texttt{.pdf}, preserving the metadata from the first notebook. See
\href{https://ipypublish.readthedocs.io/en/latest}{ipypublish} for more
info. If you want to merge notebooks and preserve cross-referencing
between them consider using
\href{https://github.com/takluyver/bookbook}{bookbook}.

\textbf{To do}

\begin{itemize}
\tightlist
\item
  Implement dynamic python markdown
\item
  Edit TeX template to include NBIS logo and other info in front page
\item
  Add xref in NBIS text.
\end{itemize}

\hypertarget{support-request}{%
\section{Support Request}\label{support-request}}

Some markdown text.\\
\textbf{Description:} (From redmine)

\textbf{Work Log} Detailed steps from contract, if possible

List:

\begin{itemize}
\tightlist
\item
  something
\item
  something else
\end{itemize}

Numbered list

\begin{enumerate}
\def\labelenumi{\arabic{enumi}.}
\tightlist
\item
  something
\item
  something else
\end{enumerate}

\hypertarget{methods}{%
\section{Methods }\label{methods}}

\hypertarget{method-section-1}{%
\subsection{Method Section 1}\label{method-section-1}}

A great method description starts here\ldots{}
\cite{rolland_proteome-scale_2014}

\hypertarget{results}{%
\section{Results }\label{results}}

\hypertarget{results-1}{%
\subsection{Results 1}\label{results-1}}

Some good results\ldots{}

\hypertarget{closing-procedures}{%
\section{Closing procedures }\label{closing-procedures}}

You should soon be contacted by one of our managers, Jessica Lindvall
(jessica.lindvall@nbis.se) or Henrik Lantz (henrik.lantz@nbis.se), with
a request to close down the project in our internal system and for
invoicing matters. If we do not hear from you within \textbf{30 days}
the project will be automatically closed and invoice sent. Again, we
would like to remind you about data responsibility and acknowledgements,
see \href{/cref\%7Bdatares\%7D}{Data Responsibilities} and
\href{/cred\%7Backnowledgements\%7D}{Acknowledgements}.

You are naturally more than welcome to come back to us with further data
analysis request at any time via
\href{http://nbis.se/support/support.html}{Support Web}. \textbf{Thank
you for using NBIS, we wish you the best of luck with your future
research!}

\hypertarget{practical-information}{%
\section{Practical Information}\label{practical-information}}

\textbf{Data responsibilities}\\
Unfortunately, NBIS does not have resources to keep any files associated
with the support request; we kindly suggest that you safely store the
results delivered by us. In addition, we kindly ask that you remove the
files from UPPMAX/UPPNEX. The main storage at UPPNEX is optimized for
high-speed and parallel access, which makes it expensive and not the
right place for long-term archiving. Please be considerate of your
fellow researchers by not taking up this expensive space.\\
The responsibility for data archiving lies with universities and we
recommend asking your local IT for support with long-term data storage.
The \href{https://www.scilifelab.se/data/}{Data Center} at SciLifeLab
may also be of help with discussing other options. Please note that
special considerations may apply to human-derived, sensitive personal
data. This should be handled according to specific laws and regulations
as outlined at the \href{https://nbis.se/support/human-data.html}{NBIS
website}.

\textbf{Acknowledgments}\\
If you are presenting the results in a paper, at a workshop or at
aconference, we kindly remind you to acknowledge us according to the
signed
\href{https://nbis.se/assets/doc/nbis-support-useragreement.pdf}{NBIS
User Agreement}:\\
NBIS staff should be included as co-authors if the support work leads to
a publication and when this is merited in accordance to the ethical
recommendations for authorship, i.e.~the
\href{http://www.icmje.org/recommendations/}{ICMJE recommendations}.

If applicable, please include the NBIS expert as co-author.

If the above is not applicable, please acknowledge NBIS like so:
\emph{Support by NBIS (National Bioinformatics Infrastructure Sweden) is
gratefully acknowledged.}

In addition, Uppmax kindly asks you to acknowledge UPPMAX and SNIC. If
applicable, please add: \emph{The computations were performed on
resources provided by SNIC through Uppsala Multidisciplinary Center for
Advanced Computational Science (UPPMAX) under Project \textbf{{[}User
project number{]}}.}

In any and all publications based on data from NGI Sweden, the authors
must acknowledge SciLifeLab, NGI and Uppmax, like so: \emph{The authors
would like to acknowledge support from Science for Life Laboratory, the
National Genomics Infrastructure, NGI, and Uppmax for providing
assistance in massive parallel sequencing and computational
infrastructure.}

\hypertarget{markdown}{%
\section{Markdown}\label{markdown}}

Below we have the definitions for mulitple figures, equations, tables,
and so on. This is kept for ease of use.

\hypertarget{todo-notes}{%
\subsection{Todo notes}\label{todo-notes}}

\todo[inline]{an inline todo}

Some text.\todo{a todo in the margins}

\hypertarget{text-output}{%
\subsection{Text Output}\label{text-output}}

\begin{lstlisting}[aboveskip=5pt,backgroundcolor=\color{blue!10},belowskip=5pt,breakindent=0pt,language={},numbers=none,postbreak={},xrightmargin=7pt]

This is some printed text,
with a nicely formatted output.

\end{lstlisting}

\hypertarget{images-and-figures}{%
\section{Images and Figures}\label{images-and-figures}}

    \begin{figure}[!bh]\begin{center}\adjustimage{max size={0.9\linewidth}{0.9\paperheight}}{Example_files/output_25_0.png}\end{center}\caption{A nice picture.}\label{fig:example}\end{figure}

    \begin{figure}[H]\begin{center}\adjustimage{max size={0.9\linewidth}{0.9\paperheight}}{Example_files/output_26_0.png}\end{center}\caption{Horizontally aligned images.}\label{fig:example_h}\end{figure}

    \begin{figure}[H]\begin{center}\adjustimage{max size={0.9\linewidth}{0.9\paperheight}}{Example_files/output_27_0.png}\end{center}\caption{Vertically aligned images.}\label{fig:example_v}\end{figure}

    \begin{figure}[H]\begin{center}\adjustimage{max size={0.9\linewidth}{0.9\paperheight}}{Example_files/output_28_0.png}\end{center}\caption{Images aligned in a grid.}\label{fig:example_grid}\end{figure}

\hypertarget{references-and-citations}{%
\subsection{References and Citations}\label{references-and-citations}}

To add citations and bibliography, use
\href{https://www.zotero.org/download/}{Zotero's Firefox plugin} and
\href{https://github.com/retorquere/zotero-better-bibtex}{Zotero Better
Bibtex}. Don't worry if the next crossreferences are not rendered in the
markdown, they are rendered on the final PDF.

References to \cref{fig:example}, \cref{tbl:example},
\cref{eqn:example_sympy} and \cref{code:example_mpl}.

Referencing multiple items:
\cref{fig:example,fig:example_h,fig:example_v}.

A single \cite{thul_subcellular_2017} or multiple latex citations
\cite{uhlen_pathology_2017, thul_subcellular_2017}

A html citation.\cite{uhlen_pathology_2017}

\hypertarget{displaying-a-plot-with-its-code}{%
\subsection{Displaying a plot with its
code}\label{displaying-a-plot-with-its-code}}

\begin{codecell}
\caption{The plotting code for a matplotlib figure (\cref{fig:example_mpl}).}
\label{code:example_mpl}
\begin{lstlisting}[language=Python,numbers=left,xleftmargin=20pt,xrightmargin=5pt,belowskip=5pt,aboveskip=5pt]
#note the "hideCode": false   in the cell's metadata
import matplotlib.pyplot as plt
plt.scatter(np.random.rand(10), np.random.rand(10),
            label='data label')
plt.ylabel(r'a y label with latex $\alpha$')
plt.legend();
\end{lstlisting}\end{codecell}

\begin{figure}[H]\begin{center}\adjustimage{max size={0.9\linewidth}{0.9\paperheight}}{Example_files/output_34_0.png}\end{center}\caption{A matplotlib figure, with the caption set in the markdowncell above the
figure.}\label{fig:example_mpl}\end{figure}

\hypertarget{displaying-a-plot-without-its-code}{%
\subsection{Displaying a plot without its
code}\label{displaying-a-plot-without-its-code}}

\begin{figure}[H]\begin{center}\adjustimage{max size={0.9\linewidth}{0.9\paperheight}}{Example_files/output_36_0.png}\end{center}\caption{A figure with hidden code}\label{fig:example_mpl2}\end{figure}

\hypertarget{tables-with-pandas}{%
\section{Tables (with pandas)}\label{tables-with-pandas}}

\begin{codecell}[H]
\caption{The plotting code for a pandas Dataframe table (\cref{tbl:example}).}
\label{code:example_pd}
\begin{lstlisting}[language=Python,numbers=left,xleftmargin=20pt,xrightmargin=5pt,belowskip=5pt,aboveskip=5pt]
df = pd.DataFrame(np.random.rand(3,4),columns=['a','b','c','d'])
df.a = ['$\delta$','x','y']
df.b = ['l','m','n']
df.set_index(['a','b'])
df.round(3)
\end{lstlisting}\end{codecell}

\begin{table}[H]
\caption{An example of a table created with pandas dataframe.}\label{tbl:example}
\centering
\begin{adjustbox}{max width=\textwidth}\rowcolors{2}{gray!20}{white}
\begin{tabular}{lllrr}
\toprule
{} &         a &  b &      c &      d \\
\midrule
0 &  $\delta$ &  l &  0.583 &  0.279 \\
1 &         x &  m &  0.914 &  0.021 \\
2 &         y &  n &  0.333 &  0.116 \\
\bottomrule
\end{tabular}

\end{adjustbox}
\end{table}

\hypertarget{equations-with-ipython-or-sympy}{%
\section{Equations (with ipython or
sympy)}\label{equations-with-ipython-or-sympy}}

\begin{equation}\label{eqn:example_ipy}
 a = b+c
\end{equation}

\begin{codecell}[H]
\caption{The plotting code for a sympy equation (\cref{eqn:example_sympy}).}
\label{code:example_sym}
\begin{lstlisting}[language=Python,numbers=left,xleftmargin=20pt,xrightmargin=5pt,belowskip=5pt,aboveskip=5pt]
f = sym.Function('f')
y,n = sym.symbols(r'y \alpha')
f = y(n)-2*y(n-1/sym.pi)-5*y(n-2)
sym.rsolve(f,y(n),[1,4])
\end{lstlisting}\end{codecell}

\begin{equation}\label{eqn:example_sympy}
\left(\sqrt{5} i\right)^{\alpha} \left(\frac{1}{2} - \frac{2 i}{5} \sqrt{5}\right) + \left(- \sqrt{5} i\right)^{\alpha} \left(\frac{1}{2} + \frac{2 i}{5} \sqrt{5}\right)
\end{equation}

\hypertarget{embed-interactive-html-like-ipywidgets}{%
\section{Embed interactive HTML (like
ipywidgets)}\label{embed-interactive-html-like-ipywidgets}}

Interactive HTML was created using ipyvolume and will render below in
.html type outputs:

% sort citations by order of first appearance
\bibliographystyle{unsrtnat}
\bibliography{Example_files/example}

\end{document}
